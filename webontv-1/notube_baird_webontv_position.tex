\documentclass[]{article}%(fold)
\usepackage[utf8]{inputenc}
\usepackage{fullpage,ifpdf,url,authblk,xspace}
\renewcommand\Affilfont{\small}

\ifpdf
\usepackage[pdftex]{graphicx}
\else
\usepackage{graphicx}
\fi
\date{}

\title{APIs and URLs for Social TV}


\author{Dan Brickley}
\affil{Vrij Universiteit, Amsterdam; NoTube Project}
\author{Libby Miller}
\author{Vicky Buser}
\affil{BBC R\&D; NoTube Project}
\author{Mo McRoberts}
\affil{Project Baird}

%(end)
\begin{document}


\ifpdf
\DeclareGraphicsExtensions{.pdf, .jpg, .tif}
\else
\DeclareGraphicsExtensions{.eps, .jpg}
\fi

\maketitle
Integration of TV and the social web is already happening - and occurring in an interesting direction. People are increasingly using online social networks to talk about TV, nearly all via second screens. This trend started without any specific tools to support it, but as TV and the Web converge, there is a risk of fragmentation of audiences for programmes over multiple applications, devices, and websites. If these silos are created, consumers, manufacturers and developers all lose out. We propose that rethinking the role of metadata as an advertisement for programmes, allowing API access to TV devices, and using URLs for identification are three techniques that would lower costs and increase creativity and thereby benefit consumers.

\section{Social networks are where the integration of Web and TV is already happening}

For broadcast TV, social networks are where the integration of Web and TV is already happening. There is evidence to suggest that a high proportion of the conversations in social media are around what people are watching on TV\footnote{For example: a YouGov/Deloitte report published in August 2010 found that 42\% of those UK adults who use the Internet while watching television do so to discuss or comment on the programmes they are watching at the time (http://today.yougov.co.uk/consumer/television-going-social). Similarly, a Twitter survey (conducted by BBC Audience Research) in August 2010 found that 49\% of UK Twitter users in the sample said they used Twitter regularly when watching TV.}. During prime-time scheduling in the UK and US, Twitter trending topics are often TV-related, and this Twitter activity can influence what people decide to watch. For example, people reported watching The Eurovision Song Contest on the basis of what was being said about it on Twitter, even though they wouldn't normally have watched it\footnote{\url{http://www.broadstuff.com/archives/1696-Eurovision-songs-sound-better-on-Twitter......html}}. 


\section{Silos are being created}

There are many new and upcoming TV or TV-like devices becoming available, for example internet-connected TVs (for example Samsung TVs with Yahoo widgets enabling you to access your social network), and set top boxes such as the Boxee Box, AppleTV and GoogleTV. 

More and more specific applications are being created, some to control various kinds of software and hardware TV devices (for example MythTV, XBMC and Boxee iPhone and Android remotes). 

Many applications are being made for specific particular programmes or events, for example Channel 4's game show `The Million Pound Drop' includes an online element that lets users play along live as the show progresses\footnote{\url{http://www.channel4.com/programmes/the-million-pound-drop-live/articles/game}}. Other examples include ITVLive during the World Cup - an experimental but very popular service\footnote{\url{http://paidcontent.co.uk/article/419-the-new-live-tv-how-real-time-social-media-are-upgrading-the-box/}} PickLive for playing along during football matches \footnote{\url{https://picklive.com/}} and the programme specific Seven Days application \footnote{\url{hhttp://sevendays.channel4.com/}}.

No one company is currently winning in all of these areas - in fact each is winning in different areas (for example, one might argue that Apple has beaten Microsoft at mobile; that Facebook has beaten Apple at social networks; Google is currently beating Apple at remote OSs; Apple is currently beating Google at set-top box, ....). The result is that silos are being created, such that people need a specific piece of hardware or software to participate in the creative applications that are being made. Nevertheless they continue to use the Web - in the guise of the social web - to talk about broadcast TV. 

\subsection{Audiences, Manufacturers, Developers and Content Owners all Lose}

Silos put barriers on participation making it more difficult for people to talk about what they are watching in a meaningful way. If potential members of the audience for a programme have to use the same hardware, or download the same application, or be on the same social network to participate, for most programmes they simply will not do it, and the potential value to them and to the rights holders in terms of  increased audiences, engaged audiences, and feedback is lost. Manufacturers and developers have to take a risk on which formats and protocols to support and reducing the time they can spend on creative solutions to consumer interests and problems. 

\section{Key problems for social TV application developers}

There are a number of common problems encountered by developers making applications for TV.
\\

{\bf{(a) ``How do we know what the person is watching?" }}

I.e. find out from the device or other means, and identify it in the wider context of large volumes of TV programmes, broadcast and on-demand, in order to do something with it, such as provide more information about the programme, connect them to other people watching it.
\\

{\bf{(b) ``How do we get extra information about the programme?" }}

Identify specific, accurate information about it such as a description, reviews,  but also and who else is watching or planning to watch. Poor information is worse than none in this case.
\\

{\bf{(c)``How do we locate apps/web pages/whatever related to it?" }}

Identify accurate related information. Lack of accuracy and specificness, as with metadata, is worse than useless.
\\

{\bf{(d) ``How can we manipulate it?"}}

Change channel, play / pause, record items for later, the usual functions of a TV.

NoTube and Project Baird have been working on various experimental and interim solutions to some of these problems, including  `NOWP' (what's playing now) (a),  TVDNS and programmes resolver (b,c)  a remote control prototcol (`Buttons') (d,a). These are workarounds for genuine problems with dealing with broadcast TV for social application developers. 

\section{Three parts of a long-term solution}

{\bf{Rethink the role of metadata}}

Use metadata as an advert for the content that can flow out into the public Web in search of viewers, rather than a precious resource to be parcelled out and sold. By making it public, and in open formats and licenses, the metadata can be a tool to draw users in via applications made by third parties.

The BBC Backstage work with TVAnytime showed that when application developers have access to this kind of data they can make very creative applications (MightyTV came about from use of the TV-Anytime data - see the Backstage e-book.  The use of the old web API based on the same data led to the BBC Radio Widget (by Phantom Gorilla) for Macs that was distributed on the CD on the front cover of Macworld). BBC's /programmes (a URL for every BBC programme) has enabled people to talk about programmes on now, even if they are not yet available on the on demoand service iPlayer. Recently, another UK broadcaster, Channel 4, has seen the benefit of having URLs suitable for sharing for their programmes).
\\

{\bf{Create URLs for the content items}}

Part of making metadata an advert for the content is helping people easily share information about it. Using an URL for a programme gives it life before, during and after broadcast, allowing interest in it to circulate through social networks. Being easily sharable It becomes something that people can link to on the social Web to allow others to understand what they are talking about. 

Once this role for metadata is understood and the programme has a unique URL, identifying a specific programme becomes trivial, and applications can add specific value. Adding resolvable {\em{machine-processible}} URLs (such as JSON or RDF) allow developers to find more information about a programme and display it suitably to the end user. Where possible, values of metadata fields should where possible also be URLs (such as Wikipedia pages, IMDB pages) so matching becomes easier (eg. against Facebook social graph LIKEs).
\\

{\bf{Agree on an open API for controlling the TV and getting access to metadata from it}}

The final piece of the puzzle is to allow other applications to access and control the TV playing device, application or web page. Whether using a second screen or looking at widgets on a TV screen, to be interesting and useful to audiences, an application needs to be able to identify what is currently playing and to allow the user to easily manipulate what they can see. So together with access to metadata, availability of URLs for content items, this needs information about what the device is showing available to other devices or applications in a well-documented, open fashion, and allow other devices to control what is being shown, change channel, play / pause, and similar conventional remote functions.


\subsection{Audiences, Manufacturers, Developers and Content Owners all Benefit}

What are the benefits of having unique URLs for programmes, open metadata, and an API to the TV?
\\

{\bf{Benefits to Audiences}}

With these features, audiences could expect diverse, attractive and accessible software and hardware remotes, and fewer remotes keep track of (physical or software). They should see many more creative applications for viewing what they may be able to watch on their TV via their phone, tablet or laptop, and simpler, faster text entry via other devices to their TVs. 

They could expect simpler access to information about programmes that they can share using their favourite social application, better programme-specific applications, and better second screen TV applications providing more interesting and relevant information about programmes, and better recommendations based on defragmented statistical information about watching behaviour.

They should be able to use data provided from social media to find interesting things to watch even when they don't themselves participate, because of the ability to identify what is being watched over large populations in real time.
\\

{\bf{Benefits to Content Producers}}

Content producers should expect a less fragmented audience, more participation and interest, and the ability to track aggregate usage of content and market that content over its lifecycle, from broadcast to on-demand to archive.
\\

{\bf{Benefits to Manufacturers and Developers}}

A single, open, well-documented API to support, rather than complex, secret multiple ones, meaning more time to focus on creative solutions to the problem of finding what to watch.

\section{Challenges and Conclusions}

To an extent is already happening. Boxee, XBMC, MythTV have http APIs to their content and Boxee makes an effort to find URLs to identify the content. People are already using social applications to talk about TV. Our three principles suggest ways in which W3C could influence the future of TV in ways that benefit audiences as well as companies involved. There are significant challenges around preserving the privacy of consumers and helping them understand what privacy risks they face, but W3C is well-placed to do this.


\end{document}
